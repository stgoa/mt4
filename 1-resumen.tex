% Resumen Tesis
% ===============
\section*{Título de la Tesis}

\subsection*{Introducción}


La cristalización de lactosa ha sido ampliamente estudiada debido al interés que genera por su extendido uso en las industrias farmacéutica y alimentaria\cite{haase1966kinetic}


\subsection*{Hipótesis}

La hipótesis de este trabajo es que el desarrollo de una técnica de video-microscopía permite cuantificar con mayor precisión la cinética de cristalización de lactosa a partir de nucleación primaria heterogénea en comparación con técnicas tradicionales como la refractometría.



\subsection*{Objetivos}
En concordancia con la hipótesis propuesta, el objetivo general de esta tesis consiste en desarrollar un método no invasivo para monitorear la cristalización de lactosa desde soluciones sobresaturadas en que ocurra sólo nucleación primaria heterogénea.



\subsection*{Metodología}

Con este fin, se prepararán soluciones acuosas de lactosa por calentamiento hasta su completa disolución. Luego, las soluciones serán vertidas en cápsulas de vidrio cubiertas con placas de vidrio y selladas con silicona para evitar la evaporación. Para el estudio de la cristalización por video-microscopía, una de las cápsulas preparadas será ubicada sobre una platina térmica que mantiene la temperatura constante


\subsection*{Resultados esperados}
Los resultados que se espera obtener son curvas de cristalización basadas en el área total de cristales en la imagen que sean comparables con aquellas obtenidas por refractometría, tasas de crecimiento de cristales individuales en función de distintos parámetros geométricos (e.g. perímetro, eje mayor, etc.), magnitud de la dispersión de estas tasas de crecimiento para cristales de un mismo experimento, valores característicos para parámetros de morfología de cristales (e.g. circularidad) y efectos de la temperatura y la sobresaturación que concuerden con lo reportado en la literatura.